%-------------------------------------------------------------------------------
% Author        : Cesar R. Urteaga-Reyesvera
% Creation date : June 29, 2017.
% Description   : First sample text file.
%-------------------------------------------------------------------------------

\section{Section 1}

  The created package highlights text in three colors: \hB{blue}, \hR{red}, and \hG{green}.  It illustrates that you can define your own commands within a package; consequently, this can be helpful when you have a lot of them, and you want to make your code clearer.

  Also, this package was created to be used as a template to make summaries.  So, you can use the following commands:

  \begin{li}
    \item \verb=\hB=: Prints the text with blue  color: \hB{example}.
    \item \verb=\hG=: Prints the text with green color: \hG{example}.
    \item \verb=\hR=: Prints the text with red   color: \hR{example}.
  \end{li}

  Once you have chosen the color, you can underline it (\hl{l}), italicize it (\hi{i}), or/and make it bold (\hb{b}); remember that the order matters.  For instance, the command \verb=\hGlib= makes the text green with all the atributes mentioned earlier: \hGlib{example}.  However, if you only wanted the text in read and bold, use \verb=\hRb=: \hRb{example}.

  In addition, you can (\hs{superscript}) or (\hu{subscript}) the text using \verb=\hs= and \verb=\hu=, respectively.

  \subsection{Subsection 1}

    \lipsum

    \subsubsection{Subsubsection 1}

      \lipsum
      \lipsum[1-2]
      \pagebreak
