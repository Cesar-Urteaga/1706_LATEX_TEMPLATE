%-------------------------------------------------------------------------------
% Author        : Cesar R. Urteaga-Reyesvera
% Creation date : June 29, 2017.
% Description   : First sample text file.
%-------------------------------------------------------------------------------

\section{Section 1}

  The created package highlights text in three colors: \hB{blue}, \hR{red}, and \hG{green}.  It illustrates that you can define your own commands within a package; consequently, this can be helpful when you have a lot of them, and you want to make your code clearer.

  Also, this package was created to be used as a template to make summaries.  So, you can use the following commands:

  \begin{li}
    \item \textbackslash hB: Prints the text with blue  color: \hB{example}.
    \item \textbackslash hG: Prints the text with green color: \hG{example}.
    \item \textbackslash hR: Prints the text with red   color: \hR{example}.
  \end{li}

  Once you have chosen the color, you can underline it (\hl{l}), italicize it (\hi{i}), or/and make it bold (\hb{b}); remember that the order matters.  For instance, the command \textbackslash hGlib makes the text green with all the atributes mentioned earlier: \hGlib{example}.  However, if you only want the text in red and bold, use \textbackslash hRb: \hRb{example}.

  In addition, you can \hs{superscript} or \hu{subscript} the text using \textbackslash hs and \textbackslash hu, respectively.\hf{In the class are defined new commands that allows you to create footnotes and images with the commands \textbackslash hf and \textbackslash img, respectively.  What is more, you can use the ``li'' environment to itemize using squares.}

  \subsection{Subsection 1}

    \lipsum[1-6]

    \subsubsection{Subsubsection 1}

      \lipsum
      \lipsum[1-2]
      \pagebreak
